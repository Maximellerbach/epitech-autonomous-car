
\documentclass{beamer}

\usepackage{graphicx}

\usetheme{Madrid}
\usecolortheme{default}

\title{01 - DeepLearning Introduction}
\author{Maxime Ellerbach}
\date{March 2023}
\begin{document}

\begin{frame}
    \titlepage
\end{frame}


\begin{frame}{Deep Learning}
    \begin{itemize}
    \item Deep learning is a subset of machine learning that involves training artificial neural networks to learn from data.
    \item It is inspired by the structure and function of the human brain.
    \end{itemize}
\end{frame}

\begin{frame}{Neural Networks}
    \begin{itemize}
    \item Neural networks are mathematical models composed of layers.
    \item Different kind of layers: Fully Connected, Convolution, LSTMs, ...
    \item The input layer receives encoded data, such as images, text, or sound, and the output layer produces a "prediction" based on the input data.
    \end{itemize}
\end{frame}

\begin{frame}{Applications}
    \begin{itemize}
    \item Deep learning is particularly effective in tasks such as:
    \begin{itemize}
    \item Image recognition
    \item Speech recognition
    \item Natural language processing
    \item Closed environment games
    \end{itemize}
    \item Traditional machine learning approaches struggle in these tasks due to the complexity and variability of the input data.
    \end{itemize}
\end{frame}

\begin{frame}{Training}
    \begin{itemize}
    \item Training a deep neural network involves feeding it with large amounts of labeled data.
    \item The network adjusts its weights and biases to minimize the error between its predictions and the true labels.
    \item This process is typically done through a technique called backpropagation which uses the chain rule of calculus to compute the gradients of the error with respect to the network parameters.
    \end{itemize}
\end{frame}

\begin{frame}{Training Approaches}
    \begin{itemize}
    \item Two main approaches to training deep neural networks: supervised and unsupervised learning.
    
    \begin{itemize}
        \item Supervised learning: network learns from labeled data by minimizing the difference between its predictions and true labels.
        \item Unsupervised learning: network discovers patterns in unlabeled data by minimizing reconstruction error or maximizing probability of the data given the model.
    \end{itemize}
    
    \item Other approaches: semi-supervised learning and reinforcement learning.
    \end{itemize}
\end{frame}


\end{document}